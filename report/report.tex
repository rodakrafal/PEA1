\documentclass[a4paper,12pt]{article}
\usepackage[utf8]{inputenc}
\usepackage{graphicx}
\usepackage{titlesec}
\usepackage{polski}
\usepackage{geometry}
\usepackage{indentfirst}
\usepackage{listings}
\usepackage{algorithm}
\usepackage{algorithmic}
\usepackage{pseudocode}

\geometry{
	a4paper,
	total={170mm,257mm},
	left=20mm,
	top=20mm,
}


\begin{document}
\begin{titlepage}
    \newgeometry{top=5.5cm, bottom=3cm}
    
    \centering
    {\huge\bfseries PEA\par}
    
    \vspace{0.9cm}
    Pierwszy projekt z Projektowania Efektywnych Algorytmów
    \vspace{1cm}
    {\Large \bfseries \\Implementacja i analiza efektywności algorytmu B\&B oraz programowania dynamicznego\par}
    \vspace{3cm}
    {\large\bfseries Rafał Rodak 252788\par}
    \vspace{0.15cm}

    \vfill

    \includegraphics[width=0.8\textwidth]{logo.png}
    \vspace{1.2cm}
    \\Prowadzący: Dr inż. Jarosław Mierzwa \\
    Termin: pt. 11:15


    \vspace{1cm}
    \today \\ \LaTeX
    
    \restoregeometry
\end{titlepage}

\tableofcontents
\pagebreak
 
\section{Wstęp}
\subsection{Opis zadania}

Zadanie polega na zaimplementowaniu oraz dokonaniu analizy efektywności algorytmu 
przeglądu zupełnego, podziału i ograniczeń (B\&B) oraz programowania dynamicznego 
(DP) dla asymetrycznego problemu komiwojażera (ATSP).\\

Problem komiwojażera \footnote{https://en.wikipedia.org/wiki/Travelling\_salesman\_problem} 
jest to zagadnienie optymalizacyjne polegające na znalezieniu minimalnego cyklu 
Hamiltona \footnote{https://en.wikipedia.org/wiki/Hamiltonian\_path} w pełnym grafie 
ważonym (pełny graf ważony to taki, w którym wszystkie krawędzi są ze sobą połączone 
i posiadają etykiety liczbowe przy krawędziach, które wyrażają wagę). Cykl Hamiltona 
jest to taki cykl w grafie, w którym każdy wierzchołek grafu został odwiedzony 
dokładnie raz w wyłączeniem pierwszego wierzchołka. \\

Dodatkowo do projektu utworzone zostało repozytorium na platformie GitHub zawierające
wszystkie pliki źródłowe wraz z komentarzami, sprawozdanie oraz pliki przeznaczone do 
testowania napisanych algorytmów, https://github.com/rodakrafal/PEA1
\subsection{Założenia}
Podczas realizacji zadania należy przyjąć następujące założenia:
\begin{itemize}
    \item[$-$] używane struktury danych są alokowane dynamicznie (w zależności od aktualnego
    rozmiaru problemu),
    \item[$-$] program umożliwia weryfikację poprawności działania algorytmu. W tym celu
    istnieć możliwość wczytania danych wejściowych z pliku tekstowego,
    \item[$-$] po zaimplementowaniu i sprawdzeniu poprawności działania algorytmu dokonano
    pomiaru czasu jego działania w zależności od rozmiaru problemu N,
    \item[$-$] dla każdej wartości N należy wygenerowano po co najmniej 100 losowych instancji 
    problemu (w sprawozdaniu należy umieścić tylko wyniki uśrednione – pamiętać, aby nie mierzyć czasu
    generowania instancji),
    \item[$-$] implementacja algorytmów została dokonana zgodnie z obiektowym paradygmatem
    programowania, 
    \item[$-$] program napisany został w wersji okienkowej, obiektowo,
    \item[$-$] kod źródłowy powinien jest komentowany,
    \item[$-$] algorytmy napisanę są w języku C++,
    \item[$-$] testy stworzonych algorytmów przeprowadzone zostały na wersji RELEASE,
    \item[$-$] istnieje możliwość ostatnio wczytanych oraz wygenerowanych danych oraz wygenerowania.
      

\end{itemize}
\vfill
\section{Teoria}
\subsection{Brute Force}
Metoda przeglądu zupełnego \footnote{https://en.wikipedia.org/wiki/Brute\-force\_search}, tzw. przeszukiwanie wyczerpujące (eng. Exhaustive serach) bądź
metoda siłowa (eng. Brute force), polega na znalezieniu wszystkich potencjalnych rozwiązań problemu 
oraz ich analizie w celu wybrania tego, które spełnia warunki zadania. Metodę tę stosuje się do 
rozwiązywania problemów, dla których znalezienie rozwiązania za pomocą innych dokładnych metod 
jest niemożliwe lub zbyt trudne. Główną wadą metody Brute Force jest to, że w przypadku wielu rzeczywistych problemów
liczba naturalnych kandydatów jest zbyt duża.

\begin{center}
    \begin{pseudocode}[ruled]{Brute Force}{P, c}
        
        c\GETS first(P)\\
        \WHILE c \not= \wedge
        \DO 
        \BEGIN
        \IF valid(P,c) \STMTNUM{1in}{st.1}\THEN output(P,c) \STMTNUM{1in}{st.2}
        \\c\GETS next(P,c) 
        \END
    \end{pseudocode}
\end{center}

Brute Force oblicza i porównuje wszystkie możliwe premutacje tras w celu określenia 
najkrótszej ścieżki. Aby rozwiązać problem TSP przy użyciu tej metody, należy obliczyć 
całkowitą liczbę tras, a następnie wyznaczyć długości każdej z nich wybierając za każdym razem
najkrótszą. Złożoność obliczeniowa algorytmu w przypadku problemu TSP przyjmuje postać wykładniczą
$O(n!) $.
\newline

Przykład działania algorytmu można opisać w oparciu podany powyżej pseudocode.
W pętli while oblicznane będą kolejne permutacje ściężki (\ref{st.1}), aż do momentu 
wyczerpania się możliwych połączeń. W przypadku znalezienia krótszej ścieżki będzie ona
zapamiętana (\ref{st.2}).

\subsection{Branch-N-Bound}
\subsection{Dynamic Programing}
Metoda ta określa ogólne podejście polegające na przekształceniu zadania 
optymalizacyjnego w wieloetapowy proces decyzyjny, w którym stan na każdym
etapie zależy od decyzji wybieranej ze zbioru decyzji dopuszczalnych.

\begin{center}
    \begin{pseudocode}[ruled]{Dynamic Programing}{G, n}
        \FOR k :=  2 \TO n \DO
        \BEGIN
        C(\{k\},k) := d_{1,k}\\
        \END\\
        \FOR s := 2 \TO n-1 \DO
        \BEGIN
        \FOR \forall \ S \in \{2,...,n\}, |S| = s \DO
        \BEGIN
        \FOR \forall \ k \in S \DO
        C(S,k) := min_{m \not= k,m\in S}
        \END
        \END
        \\\\opt := min_{k\not=1} [C(\{2,3,...,n\},k)+d_{k,1}]\\
        \RETURN{opt}
    \end{pseudocode} 
\end{center}

Metoda rozwiązywania problemów optymalizacyjnych dekomponowalnych rekurencyjnie 
na podproblemy.
\begin{enumerate}
    \item Scharakteryzowanie struktury rozwiązania optymalnego
    \item Zdefiniowanie kosztu rozwiązania optymalnego jako funkcji optymalnych 
    rozwiązań podproblemów (układ równań rekurencyjnych)
    \item Wyliczenie optymalnego kosztu metodą bottom-up 
    (rozwiązywanie podproblemów od najmniejszego na największego)
    \item Skonstruowanie rozwiązania optymalnego (na podstawie wykonanych obliczeń)
    - można pominąć ten krok jeżeli interesuje nas tylko koszt rozwiązania
    optymalnego a nie jego osiągnięcie
\end{enumerate}

Algorytm Helda - Karpa \footnote{https://en.wikipedia.org/wiki/Held\%E2\%80\%93Karp\_algorithm}, zwany również algorytmem Helda - Bellmana - Karpa
jest algorytmem służący do rozwiązywania problemu TSP. Jest on algorytmem dokładnym
opartym na programowaniu dynamicznym. Ma on złożoność czasową $O(n^22n)$ oraz 
złożoność pamięciową równą $O(nn^2)$. Jest to co prawda złożoność gorsza od wielomianowej, 
ale algorytm ten jest znacznie lepszy od algorytmu naiwnego sprawdzającego wszystkie warianty
takiego jak Brute Force.


Działanie algorytmu: załóżmy, że mamy graf liczący n wierzchołków ponumerowanych 
1, 2, …, n. Wierzchołek 1 niech będzie wierzchołkiem początkowym. Jako $d_{i,j}$
oznaczmy odległość między wierzchołkami i oraz j (są to dane wejściowe). 
Oznaczmy jako D(S, p) optymalną długość ścieżki wychodzącej z punktu 1 i 
przechodzącej przez wszystkie punkty zbioru S tak, aby zakończyć się w punkcie p 
(p musi należeć do S). Przykładowo, zapis D({2, 5, 6}, 5) to optymalna długość 
ścieżki wychodzącej z punktu 1, przechodzącej przez punkty 2 i 6, kończącej się w 
punkcie 5. Liczbę punktów w zbiorze S oznaczmy jako s. Tym co odróżnia metrodę programowania
dynamicznego od metody przeglądu zupełnego jest to, że nie musimy wyliczać odległości
poszczególnych wierzchołków, a jedynie skorzystać z wcześniej policzonych wyników. 
W przeglądzie zupełnym każdą drogę liczymy od nowa, nie wykorzystując wcześniejszych 
wyliczeń, co jest tratą czasu, której unikamy w metodzie programowania dynamicznego.

\section{Pomiary}
Do sprawdzenia poprawności działania algorytmu, wybrano następujący zostaw instancji:
\begin{itemize}
    \item[$*$] http://jaroslaw.mierzwa.staff.iiar.pwr.wroc.pl/pea-stud/tsp/
\end{itemize}
\end{document}